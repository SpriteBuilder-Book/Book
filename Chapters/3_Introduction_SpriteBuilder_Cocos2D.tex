\chapter{Introduction to \SB{} and \cocos{} }
Now it's time to dive into 2D Game Development! For this chapter I will assume
that you haven't written a game with a game engine so I will explain the
relevant concepts fairly detailed.

\section{Introduction to \cocos{}}
To understand what \cocos{} is, it is helpful to look at the history of game
development. Back in 1980 video games were written in assembler and images were
drawn to the screen by manually setting colors for certain pixels. Since then a
wealth of frameworks and libraries has been written to make the life of a game
developer easier. 

%stub
When working with a 2D game engine for the first time you will be introduced to
a whole set of new terminology. Just as a framework to write desktop
applications knows the concept of windows, buttons and mouse clicks a 2D game
engine comes with its own set of terms and techniques. Let's get started by
talking about \textit{Scenes}.
%stub
\subsection{Scenes}
Scenes are a term you are most likely to now from movies. However, the term is
very commonly used for game engines, too. In most game engines, including
\cocos{}, scenes are used to structure your game.

In \cocos{} scenes are the highest level on which you can structure your game
content. A common example would be game with following scenes:
\begin{itemize}
  \item Main Menu Scene
  \item Gameplay Scene
  \item Leaderboard Scene
  \item Setting Scene
\end{itemize}
By default every scene in \cocos{} is a full-screen scene. 
\subsection{Nodes}
\subsection{Scene Graphs}

\section{Introduction to \SB{}}
As of this writing SpriteBuilder itself is a very new tool, released in early
2014.

\subsection{The Editor}
\subsection{CCB Files}
\subsection{Publishing - How \SB{} and \xcode work together}
\subsection{How \SB{} and \cocos{}}
\subsection{Code Connections}

\section{A first \SB project} 