\chapter{Persisting Highscores}
The game we have built so far is clearly fun to play, however, if we want
players to come back regularly we will need a highscore feature. Highscores will
motivate players to improve their skills by playing the game frequently.

For this game we will keep the mechanism very simple. For each game mode we will
store the highest score the player has achieved. These scores will only be
stored locally on the device. Then we'll extend the game over popup to show the
current highscore and to inform the player in case she has beaten her old one.

\section{Extending the GameModeDelegate}
We'll start implementing this feature by extending the definition of the
\inlinecode{GameModeDelegate}. We'll add a required method called
\inlinecode{saveHighscore}. That method will be called from within
\inlinecode{MainScene} as soon a game ends. All game modes will implement this
method and perform the highscore storing code within it.

\begin{leftbar}
Open \filemention{GameModeDelegate.swift} and add the following method to the
end of the protocol definition:
\begin{lstlisting}
/**
Should be invoked when the receiving Game Mode should store the latest Highscore
*/
func saveHighscore()
\end{lstlisting}
\end{leftbar}

Now that we've extended the definition of the protocol, let's add a method call
to \inlinecode{MainScene} that uses the new \inlinecode{saveHighscore} method.
Whenever a game ends, we want to store the latest highscore of the current game
mode.

\begin{leftbar}
Extend the \inlinecode{gameOver} method in \filemention{MainScene.swift} to look
as following:
\begin{lstlisting}
func gameOver() {
  isGameOver = true
  userInteractionEnabled = false
  gameModeDelegate?.saveHighscore()
  presentGameOverPopup()
}
\end{lstlisting}
\end{leftbar}

This change was simple. Now let's implement the highscore saving mechanism for
both game modes.

\section{Storing highscores for the endless game mode}
iOS provides us a variety of options to persist application data. Core
Data offers a feature-rich object persistence API that allows for advanced features
such as search and migration between different versions of a data model. Through
\inlinecode{NSKeyedArchiver} we are able to serialize objects and store them
in files. Another option, preferred for simple tasks, is using the
\inlinecode{NSUserDefaults} class to persist information.

\inlinecode{NSUserDefaults} is a persistent key-value store with a very simple
API. We can store an integer with the following call:
\begin{lstlisting}
NSUserDefaults.standardUserDefaults().setInteger(20, forKey: "highscore")
\end{lstlisting}

And retrieving the information is just as straightforward:
\begin{lstlisting}
let oldHigschore =
NSUserDefaults.standardUserDefaults().integerForKey(highscoreKey)
\end{lstlisting}

Since we only want to score one integer per game mode, this simple API is ideal
for our purposes.

Let's start with the implementation. First we'll define a new variable and a
new constant. We'll use a variable called \inlinecode{newHighscore} to store
whether or not the latest achieved score was a highscore. Based on this variable
we will display a slightly different message to the user later on.

We'll also define a constant for the \textit{key} that we use to store and
retrieve the highscore from \inlinecode{NSUserDefaults.}

\begin{leftbar}
Add the following two member definitions to \filemention{EndlessGameMode.swift}:
\begin{lstlisting}
private let highscoreKey = "EndlessGameMode.Highschore"
private var newHighscore = false
\end{lstlisting}
\end{leftbar}

When defining a key for working with \inlinecode{NSUserDefaults} it's good
practice to prefix it with the current class name. That avoids conflicts between
different parts of your app that might store and access information in the user
defaults.

Now we can implement the \inlinecode{saveHighscore} method. We'll check if
the latest score is higher than the current highscore, if that's the case we
will persist the latest score and set the \inlinecode{newHighscore} variable to
\inlinecode{true}. Otherwise we'll simply set \inlinecode{newHighscore} to
\inlinecode{false}.

\begin{leftbar}
Add the following method to \filemention{EndlessGameMode.swift}:
\begin{lstlisting}
func saveHighscore() {
  let oldHigschore = NSUserDefaults.standardUserDefaults().integerForKey(highscoreKey)

  if (Int(survivalTime) > oldHigschore) {
    // if this score is larger than the old highscore, store it
    NSUserDefaults.standardUserDefaults().setInteger(Int(survivalTime), forKey: highscoreKey)
    NSUserDefaults.standardUserDefaults().synchronize()
    newHighscore = true
  } else {
    newHighscore = false
  }
}
\end{lstlisting}
\end{leftbar}

Now we are conforming to the new \inlinecode{GameModeDelegate} protocol and are
successfully storing new highscores! One interesting line that we did not
discuss yet is the following:
\begin{lstlisting}
NSUserDefaults.standardUserDefaults().synchronize()
\end{lstlisting}
This line forces \inlinecode{NSUserDefaults} to write the latest changes to disk
immediately. This method is called periodically by default. If we however store
more or less sensitive information, such as the latest highscore a player just
achieved, we call the method explicitly. That way the changes are persisted
right away, eliminating the risk of losing data if the app crashes or is quit by
the user.

Now there's a last step left. We should change the highscore message that we are
displaying at the end of the game to include the player's highscore. Further, if
the player just beat her own highscore we want to display a special message to
congratulate the player.

\begin{leftbar}
Replace the existing \inlinecode{highscoreMessage} method with the following
one:
\begin{lstlisting}
func highscoreMessage() -> String {
  let secondsText = "second".pluralize(survivalTime)

  if (!newHighscore) {
    let oldHighscore = NSUserDefaults.standardUserDefaults().integerForKey(highscoreKey)
    let oldHighscoreText = "second".pluralize(oldHighscore)
    
    return "You have survived \(Int(survivalTime)) \(secondsText)! Your highscore is \(Int(oldHighscore)) \(oldHighscoreText)."
  } else {
    return "You have reached a new highscore of \(Int(survivalTime)) \(secondsText)!"
  }
}
\end{lstlisting}
\end{leftbar}

One of the first things you might notice is that we've introduced a
\inlinecode{pluralize} method on \inlinecode{String}. This method is
part of the helpers that we've included right at the beginning of this project.
Since we now have multiple occasions in which we need to use the pluralized form
of a word it makes sense to factor this functionality out and avoid code
duplication. This \inlinecode{pluralize} method is very primitive, it will
append and \textit{s} to a word in case the integer passed to the method is
larger than one. That is obviously not the correct way to pluralize all English 
words, but for our game in which we use \textit{points} and \textit{seconds} it
works just fine.

In the first line of this method we determine whether we need to use the word
\textit{second} or \textit{second\textbf{s}} for this highscore message. Since
we need this part of the message in any case, we keep it outside of the
\inlinecode{if} statement.

Next, we check if the player has achieved a new highscore. If not, we display
the latest score along with the current highscore. Else, we let the player know
that he just reached a new highscore.

And this is all it takes to build a simple highscore system -
\inlinecode{NSUserDefaults} can go a pretty far way when storing this kind of
simple information.

All that is left for this chapter is adding the same highscore functionality to
the timed game mode.

\section{Storing highscores for the timed game mode}
