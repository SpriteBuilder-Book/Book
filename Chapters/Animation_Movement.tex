\chapter{Animations and Movements}
We have multiple different ways to add activity to scenes in \cocos{}. First, it
is important to know two different forms of activity:
 
\begin{description}
  \item[Movement] If the position of one of our nodes changes over time, we
  consider this a movement.
  \item[Animation] When we iterate through a set of different images, we call
  this an animation.
\end{description}

Let's take a look at movements first.

\section{Implementing Movements}
Since we are using \spriteb{}, we have two different ways to implement this:
In code or through \spriteb{}'s keyframe animations. Implementing physics for
your game is another way to add movement to Nodes, but we are going to spare
that one for later.

\subsection{Implementing Movement manually in code}
We will first try this the hard way - which is always the best way to learn
something new. Since I assume that many readers are new to game development, we
are going to take a small step back and take look at the bigger picture.

One core concept of games is the \textit{game loop}. The game loop takes care of
giving any object in the game, to implement a time based behaviour, by calling
certain methods in a regular interval. In \cocos{} the game loop by itself is
not visible, the only aspect of it we use, is an \textit{update} method. The
update method is called every render cycle. Any CCNode can override this update
method.

This update method is where we can implement any time-based actions. The method
signature looks like this:

\begin{lstlisting}
-(void)update:(CCTime)delta {
	//implement any custom movement/animation here
}
\end{lstlisting}

The one parameter we get passed in is the time that has passed since the update
method has been called last. Whichever action we perform/trigger in the update
method, we need to consider this time factor.

Let's once again implement this by example. We now want to move a simple
unanimated Sprite over the screen with a constant speed:

\begin{lstlisting}
-(void)update:(CCTime)delta {
	//TODO: implement movement code
}
\end{lstlisting}

\subsection{Implementing Movement using CCActions}
When developing games we are mostly confronted with very similar problem sets.
\cocos{} provides a lot of functionality for common use cases. One of these
convinience concepts are \textit{CCActions}. CCNodes can run CCActions. The
CCActions a CCNode runs can affect different properties of the CCNode such as
position, color or scale.

So for the use case we have seen above, moving a sprite across the screen, we
don't need to implement custom movement code, we can use CCActions.

There are a large amount of CCAction types in \cocos{}:
\begin{description}
  \item[CCActionMove] ...
\end{description}

We will first take a look at how we can implement the movement using a
CCActionMove, then we will take a clooser look at the other CCAction types and
how and when they can be used.

Actually, implementing this is amazingly easy. We create a CCActionMove and let
our CCNode run this action:
\begin{lstlisting}
	CCActionMove *move = [CCActionMove actionWithTargetPosition:pos duration:1.f];
	[ship runAction:move];
}
\end{lstlisting}

\subsection{Implementing Movement with \spriteb{}}

\section{Implementing Animations}
This section will have to challenges. First we will need to learn a new tool, to
create images in a way, that we can use them for sprite animations. Second we
will learn how to implement these animations in code and in \spriteb{}.

In \cocos{}, as in many 2D game engines, animations consist of a group of
frames. For example a running animation will be a set of 5 different pictures
showing a character in 5 different stages of a running movement. Basically all a
game has to do is to switch between these 5 different images with a certain
delay that makes the animation believable. This is the same approach as used by
GIF-File animations (these things that splattered the first generation of the
world wide web).

\section{Creating an animation Spritesheet}
\section{Implementing animations in code}
\subsection{Implementing an animation using CCActions}
\subsection{Implementing an animation using a convenience class}
\section{Implementing animations using \spriteb{}}

