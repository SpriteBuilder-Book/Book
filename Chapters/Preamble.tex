\chapter{Preample}
Welcome to \textbf{the} complete guide to iOS game programming.
You will be guided through absolutely everything you need to know about
\cocos{} and \spriteb{} and 2D game programming in general.

While we will cover the very basics of game programming, such as scene graphs,
animations and game loops - Objective-C, the language we will be using
throughout the book is not in the scope of things you will learn. When starting
this guide, you are expected to have a solid foundation of Objective-C
knowledge.

The structure in which you will learn is the following:
\begin{itemize}
  \item Tools: Get familiar with the very basics of \cocos{} and \spriteb{}
  \item Infrastructure: Understand that on a high level a game consists of
  scenes. Understand how to create scenes and navigation paths through these
  scenes with \cocos{} and \spriteb{}
  \item Action and Movement: Understand how objects in your game can be moved
  and animated. With \cocos{} and \spriteb{}
  \item Interaction: Understand how user interaction can be captured, including
  Touch interaction and Accelerometer.
  \item Interobject Interaction: Understand how to use the delightfully
  integrated Chipmunk physics engine
  \item Beyond the Basics; Recipes and Best Practices:  Once we have the basics,
  we will look at a ton of recipes and exciting \cocos{} classes, which you can
  use to create any kind of 2d game. Particle Effects, Custom Drawing, Custom
  Shaders, Tile Maps, Networking, Audio, cocos2d UI in depth, etc.
\end{itemize}

\section{Structure of this book}
This book shall function as a learning guide and a reference book. Therefore
most examples will be small and self-contained. Instead of builind a game
throughout the whole book, you will learn by implementing very small projects
that are limited to the material we are currently discussing. 
That shall give you a better chance of understanding the concepts/code
snippets and using them in your original game, instead starting of from an
example game you have built in this book.

After we have discussed all the basics and you have a good understanding of the
\cocos{} API I will point you to resources that provide example implementations
for specific game types.

There are two different ways to read this book. From the front to the beginning,
gaining knowledge in logical groups. Or if you aren't a beginner and would like
to use this book as an example driven extension of the API reference you can
look up pages by Class names or concept names. There is a special glossar in the
back of this book.

\section{Tools used throughout this book}
The two main tools we will be using are \cocos{} and \spriteb{}. Many of the
problems that occur during game development can be solved by both of these
tools. Wherever it makes sense I will point out both ways, one using only
\cocos{} and one using \spriteb{}. This will allow you to see the advantages of each approach
and finally decide which tool you want to use in certain situations for your own
games.

\section{What is a 2D game engine?}
To understand what game engines are, it is helpful to look at the history of
game development. The first video games were written in assembler (a very low level
programming language) and images were drawn to the screen by manually setting
colors for certain pixels. Since then a wealth of frameworks and libraries has been written to make the life of a game
developer easier. 