\chapter{Working with Sprites}
One of the most used classes in your games will be CCSprites. A CCSprite is a
CCNode Subclass that represents an image (for example a game character, a tree,
etc.). 
\section{Getting started with CCSprite}
The easiest way to initialize a sprite it this:
\begin{lstlisting}
-(void)update:(CCTime)delta {
	//implement any custom movement/animation here
}
\end{lstlisting}


\section{Working with Sprites}\label{working-with-sprites}

Sprites are a very important part of Cocos2D, they are used to render
textures to the screen. This documentation will cover following aspects:

\begin{itemize}
\itemsep1pt\parskip0pt\parsep0pt
\item
  Creating textured sprites
\item
  Changing a sprite texture
\item
  Sprite Frame Animations
\item
  Using 9 Patch Sprite Images
\end{itemize}

\subsection{Creating a textured
Sprite}\label{creating-a-textured-sprite}

The easiest way to setup a textured Sprite is the following line of
code:

\begin{verbatim}
CCSprite *hero = [CCSprite spriteWithImageNamed:@"hero.png"];
\end{verbatim}

\texttt{spriteWithImageNamed:} utilizes \texttt{CCFileUtils} to
automatically find the image in the right resolution, you do not have to
specify the resolution manually. The line above also works for images
contained in \emph{Smart Sprite Sheets} created by SpriteBuilder. In
case you are using folders in SpriteBuilder you need to include the
complete path to the image, e.g.:

\begin{verbatim}
CCSprite *hero = [CCSprite spriteWithImageNamed:@"gameAssets/hero.png"];
\end{verbatim}

The content size of the sprite will automatically be the size of the
texture.

\subsection{Changing the texture of an existing
Sprite}\label{changing-the-texture-of-an-existing-sprite}

To change the texture of an initialized Sprite the \emph{spriteFrame}
property can be used:

\begin{verbatim}
hero.spriteFrame = [CCSpriteFrame frameWithImageNamed:@"hero_dead.png"];
\end{verbatim}

\subsection{Sprite frame animations}\label{sprite-frame-animations}

Sprite frame animations are used in basically every 2D game. A sequence
of different 2D images gets replaced at a defined interval forming an
animation:

\begin{figure}[htbp]
\centering
\includegraphics{animation.png}
\caption{image}
\end{figure}

Sprite frame animations can be created in a plist-format which is used
by plain Cocos2D games or within SpriteBuilder.


\begin{lamp}[frametitle={Optimizing Performance with Batch Nodes}]
Batch Nodes can be used to draw many Sprites witht the same textures at once.
\end{lamp}