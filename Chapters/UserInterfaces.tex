\chapter{User Interfaces in \cocos{}}
Similar to UIKit, \cocos{} provides a set of UI components you should be
familiar with:
\begin{itemize}
  \item CCButton
  \item CCTextField
  \item CCTableView
\end{itemize}

\section{Positioning and Layouting in \cocos{}}
We will first look at positioning: a Node is placed at a certain positioning.
After that we will take a look at layouting: a Node's position is determined by
a layouting mechanism.
\subsection{Positioning}
In order to create your first menu in \cocos{} you need to understand how the
layouting/positioning within CCScenes works. If you are already
familiar with UIKit (Apple's Framework to build iOS Apps with the native
interface components) there is one significant difference: \cocos{}'s root point
(x=0, y=0) is in the bottom left corner.

Another aspect to positioning which isn't used by many other frameworks
are anchor points. Anchor points define which position within your Node is used
to place the Node. Most Node's anchor point default to 0,0 (the bottom left
corner), while some CCNode subclasses, such as CCSprite override this property.
CCSprite's anchor point defaults to (0.5, 0.5) the middle of itself.
Changing the anchor point will affect positioning and rotation of your Nodes.

\cocos{} 3.0 also introduces a new concept called \textit{positioning types}.
These positioning types influence the way, the position property of your Node is
interpreted. Here an overview of the available position types:

\begin{description}
  \item[CCNormalizedPosition] Normalized means that the position is expressed
  relative to the parents content size. A 0.5 value for the x-Position means,
  that the Node will be placed at 50 percent of the parents width - thus
  horizontally centered.
  \item[CCAbsolutePosition] Default. Interpretes the position as absolute
  position.
\end{description}

Let's come up with some layouting examples to make these theoretical terms into
practical code:

\begin{lstlisting}
/* position CCNode at top right; There are other solutions, but this is the
ideal one:*/
CCNode *node = [CCNode node];
node.anchorPoint = (1.f, 1.f)
node.positioningType = CCTypeNormalized;
node.position = (1.f,1.f);

/* Place this in the center:*/ 
CCNode *node = [CCNode node];
node.anchorPoint = (0.5f, 0.5f)
node.positioningType = CCTypeNormalized;
node.position = (0.5f,0.5f);
\end{lstlisting}

Positioning types can also be combined. Assume we want to center an element
horizontally, but want it's vertical position to be 100 points from the top of
the screen. This means we'd like normalized position for our x positioning, but
absolute positioning for our y position:

\begin{lstlisting}
/* position CCNode at top right; There are other solutions, but this is the
ideal one:*/
CCNode *node = [CCNode node];
node.anchorPoint = (0.5f, 1.f)
node.positioningType = CCPositionTypeMake(CCTypeNormalized, CCTypeAbsolute);
node.position = (0.5f,parent.contentSize.height - 100);
\end{lstlisting}
 
\subsection{Layouting}
CCLayoutBox has already been mentioned as one of the important CCNode types in
\cocos{}. CCLayoutBox can take care of automatically positioning your nodes, by
aligning them vertically or horizontally with a margin between the items that
can be defined manually.

Here's a short example:
\begin{lstlisting}
/* TODO: layout box example code */
CCLayoutBox *layoutBox = [CCLayoutBox layoutBox];
\end{lstlisting}
 
